\section{Task 2}
\begin{frame}[fragile]
\frametitle{Sequential Optimization: Matrix Access}
\textbf{Access patterns into the Matrix}
\begin {itemize}
\item Before Optimization: column wise access in relax\_jacobi
\item low spatial locality

\begin{lstlisting}
for (j = 1; j < sizex - 1; j++) {
	for (i = 1; i < sizey - 1; i++) {
		utmp[i * sizex + j] = 0.25 * (
				u[i * sizex + (j - 1)] +  // left
				u[i * sizex + (j + 1)] +  // right
				u[(i - 1) * sizex + j] +  // top
				u[(i + 1) * sizex + j]); // bottom
				
				...
	}
}
\end{lstlisting}

\end {itemize}
\end{frame}

\begin{frame}[fragile]
\frametitle{Sequential Optimization: Matrix Access}
\begin {itemize}
\item{Solution: change loop, row wise access }
\begin{lstlisting}
for (j = 1; j < sizex - 1; j++) {
	for (i = 1; i < sizey - 1; i++) {
		utmp[i + j * sizey] = 0.25 * (
			a[i + (j - 1)*sizey] +  // left
			a[i + (j + 1)*sizey] +  // right
			a[(i - 1) + j*sizey] +  // top
			a[(i + 1) + j*sizey]); // bottom
						
			...
	}
}
\end{lstlisting}

\end {itemize}
\end{frame}

\begin{frame}[fragile]
\frametitle{Sequential Optimization: No Copy}

\textbf{Avoid copy operations}
\begin {itemize}
\item Before Optimization: in relax\_jacobi, copy back all new values back to the array
\begin{lstlisting}
for (j = 1; j < sizex - 1; j++) {
	for (i = 1; i < sizey - 1; i++) {
		u[i * sizex + j] = utmp[i * sizex + j];
	}
}
\end{lstlisting}

\end {itemize}
\end{frame}

\begin{frame}[fragile]
\frametitle{Sequential Optimization: No Copy}
\begin {itemize}

\item Idea: Instead of recomputing, just swap the pointers
\begin{lstlisting}
void arraySwap(double** a, double** b, int sizex, int sizey) {
	double *temp = *a;
	*a = *b;
	*b = temp;
}
\end{lstlisting}

\end {itemize}
\end{frame}

\begin{frame}[fragile]
\frametitle{Sequential Optimization: No Copy}
\begin {itemize}

\item Important: Need to initialize uhelp with u 
\begin{lstlisting}
	for (i = 0; i < np; i++) {
		for (j = 0; j < np; j++) {
			(param.uhelp)[i * np + j] = (param.u)[i * np + j];
		}
	}
\end{lstlisting}
\item Note: this took us 2 days to figure out

\end {itemize}
\end{frame}

\begin{frame}[fragile]
\frametitle{Sequential Optimization: Residual}

\textbf{Calculation of Residuum}
\begin {itemize}
\item Before Optimization: For residual at timestep t, compute entire relaxation of timestep t+1
\begin{lstlisting}
for (j = 1; j < sizex - 1; j++) {
	for (i = 1; i < sizey - 1; i++) {
		unew[i + j * sizey] = 0.25 * (
			a[i + (j - 1)*sizey] +  // left
			a[i + (j + 1)*sizey] +  // right
			a[(i - 1) + j*sizey] +  // top
			a[(i + 1) + j*sizey]); // bottom
						
			diff = unew - u[i * sizex + j];
			sum += diff * diff;
	}
}
\end{lstlisting}

\end {itemize}
\end{frame}

\begin{frame}[fragile]
\frametitle{Sequential Optimization: Residual}
\begin {itemize}

\item Idea: Save results of previous timestep to compute residual
\begin{lstlisting}
for (j = 1; j < sizex - 1; j++) {
	for (i = 1; i < sizey - 1; i++) {
		diff = utmp[i+j*sizey] - u[i+j*sizey];
		sum += diff * diff;
	}
}
\end{lstlisting}

\end {itemize}
\end{frame}

\begin{frame}[fragile]
\frametitle{Sequential Optimization: Residual}
\begin {itemize}

\item Second Idea: Compute relaxation directly with residual and save for later
\begin{lstlisting}
for (i = 1; i < sizey - 1; i++) {
	for (j = 1; j < sizex - 1; j++) {
		atmp[i * sizex + j] = 0.25 * (
			a[i * sizex + (j - 1)] +  // left
			a[i * sizex + (j + 1)] +  // right
			a[(i - 1) * sizex + j] +  // top
			a[(i + 1) * sizex + j]); // bottom
			
			diff = atmp[i * sizex + j] - a[i * sizex + j];
			sum += diff * diff;
	}
}
return sum;
\end{lstlisting}

\end {itemize}
\end{frame}

\begin{frame}[fragile]
\frametitle{Sequential Optimization: Interleaving}


\textbf{Cache optimization by interleaving of iterations: tiling and wavefront}
\begin {itemize}
\item Due to the spatial dependencies between the grid points, tiling and wavefront become non-trivial
\item Traditional tiling would yield wrong/old values at the borders
\item Idea: Tile grid and compute one iteration on tiles
\item However: Due to the already sequential access pattern, wavefront interleaving is not expected to give a significant improval of cache behaviour in a non-parallel environment

\end {itemize}
\end{frame}

\begin{frame}[fragile]
\frametitle{Sequential Optimization: Interleaving}

\begin {itemize}
\item Tiling Solution:

\begin{lstlisting}
for (tiley = 1; tiley < sizey-1; tiley += tilesize){
	for (tilex = 1; tilex < sizex-1; tilex += tilesize){
		for (i = tiley; (i < sizey - 1) && (i < tiley+tilesize); i++) {
			for (j = tilex; (j < sizex - 1) && (j < tilex+tilesize); j++) {
				atmp[i * sizex + j] = ...
			}
		}
	}
}
\end{lstlisting}
\end {itemize}
\end{frame}

\section{Task 2.1}
\begin{frame}
\frametitle{Task 2.1 Sequential optimization - MFlop/s}
	\begin{tikzpicture}
		\begin{axis}[
			legend style={cells={align=left}},
			xlabel=resolution,
			ylabel=(PAPI) MFlop/s,
			title=Performance of Jacobi with various optimization steps (all -O2),
			legend pos = outer north east,
			no markers]
			\addplot[color=red] table {plots/jacobi_v2/0_va_flops.dat};.
			\addplot[color=green] table {plots/jacobi_v2/2_ca_flops.dat};.
			\addplot[color=blue] table {plots/jacobi_v2/3_ps_flops.dat};.
			\addplot[color=magenta] table {plots/jacobi_v2/4_aio_flops.dat};.
			\addplot[color=cyan] table {plots/jacobi_v2/5_tile_flops.dat};.
			\legend{-vanilla, -cache\\ aligned, -pointer\\ swap, -all in one, -tiled}
		\end{axis}
	\end{tikzpicture}
\end{frame}

\begin{frame}
\frametitle{Task 2.1 Sequential optimization - MFlop/s}
	\begin{tikzpicture}
		\begin{axis}[
			legend style={cells={align=left}},
			xlabel=resolution,
			ylabel=(PAPI) MFlop/s,
			title=PAPI vs hardcoded (iter x 11 flops x grid²),
			legend pos = outer north east,
			no markers]
			\addplot[color=red] table {plots/jacobi_v2/4_aio_flops.dat};.
			\addplot[color=green] table {plots/jacobi_v2/4_aio_hard_flops.dat};.
			\legend{-PAPI, -hard}
		\end{axis}
	\end{tikzpicture}
\end{frame}

\section{Task 2.2}
\begin{frame}
\frametitle{Task 2.1 Sequential optimization - L2 Missrate}
	\begin{tikzpicture}
		\begin{axis}[
			legend style={cells={align=left}},
			xlabel=resolution,
			ylabel=(PAPI) L2 Cache Missrate,
			title=Performance of Jacobi with various optimization steps (all -O2),
			legend pos = outer north east,
			no markers]
			\addplot[color=red] table {plots/jacobi_v2/0_va_cache.dat};.
			\addplot[color=green] table {plots/jacobi_v2/2_ca_cache.dat};.
			\addplot[color=blue] table {plots/jacobi_v2/3_ps_cache.dat};.
			\addplot[color=magenta] table {plots/jacobi_v2/4_aio_cache.dat};.
			\addplot[color=cyan] table {plots/jacobi_v2/5_tile_cache.dat};.
			\legend{-vanilla, -cache\\ aligned, -pointer\\ swap, -all in one, -tiled}
		\end{axis}
	\end{tikzpicture}
\end{frame}


\begin{frame}[fragile]
\frametitle{Task 2.2: Power of Two}

\textbf{ There is a performance issue around sizes of powers of 2, e.g. np=1020/1022/1024/1026.}
\begin {itemize}
\item Can you explain it? 
\item How to get rid of that issue?
\end {itemize}

When using array sizes of powers of two, the entries above and below will have similar least significant bits, this will cause them to be put into the same cache set. Hence, conflict misses will become more probable.

A way to get rid of this problem is array padding, so padding the arrays with a couple more entries, so that the least significant bits of neighboring cachelines differ, therefore reducing conflict misses.


\end{frame}
