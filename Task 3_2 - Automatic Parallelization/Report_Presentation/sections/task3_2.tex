\section{Task 3.2}
\begin{frame}[fragile]
\frametitle{Task 3.2 Automatic Parallelization - Getting it to work}
\begin{itemize}
\item flag \texttt{-parallel}: enables auto parallelization
\item flag \texttt{-opt-report-phase=par}: compilation yields report about parallelized parts, parts that could not be parallelized and the underlying reasons
\item BUT: reports no loops were changed due to assumed data dependencies
\item \texttt{\#pragma ivdep}: ignores dependencies for a single loop
\item BUT: still no parallelization. Turns out icc has a threshold for optimizations it thinks are not worth it. 
\item flag \texttt{-par-threshold0}: modifies (0=removes) this threshold
\item alternatively \texttt{\#pragma parallel always}: overrides threshold for a single loop 
\item flag \texttt{-guide-par}: returns possible manual optimizations during compilation\; did not find anything for relax\_jacobi.c 
\end{itemize}
\end{frame}

\begin{frame}[fragile]
\frametitle{Task 3.2 Automatic Parallelization - pragma snippet}
\begin{lstlisting}
double relax_jacobi( double **u1, double **utmp1,
         unsigned sizex, unsigned sizey )
{ [...]
@#pragma parallel always@
  for( i=1; i<sizey-1; i++ ) {
  	int ii=i*sizex;
  	int iim1=(i-1)*sizex;
  	int iip1=(i+1)*sizex;

@#pragma ivdep@
    for( j=1; j<sizex-1; j++ ){
       unew = 0.25 * (u[ ii+(j-1) ]+
        		            u[ ii+(j+1) ]+
        		            u[ iim1+j ]+
        		            u[ iip1+j ]);
		    diff = unew - u[ii + j];
		    utmp[ii+j] = unew;
		    sum += diff * diff;

    }
  } [...]
}
\end{lstlisting}
\end{frame}

\begin{frame}
\frametitle{Task 3.2 Automatic Parallelization - Speedup 1/2}
\begin{itemize}
\item Do performance measurements for 1, 2, 4, 8, 16 and 32 threads on SuperMUC [...] for problem size 5200. [...] sequential code as the basis for the speedup calculation. \\
\textbf{See next slide.}
\item Is there a difference between the sequential time from Assignment 3.1 and the sequential time of the OpenMP version? \\
\textbf{Slight degradation in performance, likely due to parallelization overhead.}
\end{itemize}
\end{frame}

\begin{frame}
\frametitle{Task 3.2 Automatic Parallelization - Speedup 2/2}
	\begin{tikzpicture}[every node near coord/.style={anchor=\alignment}]
		\begin{axis}[
			legend style={cells={align=left}},
			xtick={1,2,4,8,16,32},
			xlabel=\#threads,
			ylabel=speedup vs sequential execution,
			title=Speedup of icc auto parallelization,
			legend pos = outer north east,
			nodes near coords,
			no markers]
			\addplot[color=red] table {plots/res_speedups.out};.
		\end{axis}
	\end{tikzpicture}
\end{frame}

\begin{frame}
\frametitle{Task 3.2 - Curious performance with many threads}
	\begin{tikzpicture}
		\begin{axis}[
			legend style={cells={align=left}},
			xtick={8,16,32},
			xlabel=\#threads,
			ylabel=PAPI MFlop/s,
			title=Mflop/s for 8/16/32t for different runs at 5200 res,
			legend pos = outer north east,
			nodes near coords,
			scaled ticks = false,
			y tick label style={
        		/pgf/number format/.cd,
		            fixed,
        		    fixed zerofill,
		            precision=0,
        		/tikz/.cd
		    },
			no markers]
			\addplot[color=red] table {plots/run1_5200res_8-16-32.out};.
			\addplot[color=green] table {plots/run2_5200res_8-16-32.out};.
			\addplot[color=blue] table {plots/run3_5200res_8-16-32.out};.
			\legend{1st run, 2nd run, 3rd run}
		\end{axis}
	\end{tikzpicture}
\end{frame}

\begin{frame}
\frametitle{Task 3.2 - Curious performance with many threads}
\begin{itemize}
\item Performance with 8 to 32 threads tends to fluctuate. 
\item Possible explanation: \begin{itemize}
\item We run on Intel Xeon Processor E5-2680 CPUs, each has 8c/16t, thus any run with \textgreater 8 threads uses Hyper-Threading
\item "Computational intensive applications with fine-tuned floating-point operations have less chance to be improved in performance from Hyper-Threading, because the CPU resources could already be highly utilized."
\item "Cache-friendly applications might suffer from Hyper-Threading enabled, because logical processors share the caches and thus the processes running on the logical processors might be competing for the caches' access, which might result in performance degradation."
\item Source: \url{https://www.researchgate.net/publication/267242498_An_Empirical_Study_of_Hyper-Threading_in_High_Performance_Computing_Clusters}
\end{itemize} 
\end{itemize}
\end{frame}